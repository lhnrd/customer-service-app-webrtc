% ----------------------------------------------------------
% Introdução (obrigatório)
% ----------------------------------------------------------
\chapter[Introdução]{Introdução}\label{chap:introducao}

Com a crescente competitividade e concorrência de mercado, uma das maneiras de um negócio se destacar e conseguir sobreviver nesse realidade é investir no relacionamento com o cliente. Esse investimento busca melhorar o grau de satisfação do cliente, que possui um papel muito importante na longevidade do negócio.

Podemos identificar alguns problemas que afetam empresas no mercado atual em relação a este assunto:
\begin{itemize}
\item Empresas estão falhando em conhecer com precisão o perfil de seus consumidores; 
\item Há grande preocupação em adquirir novos clientes, mas não em retê-los. De acordo com \cite{marketingmetrics2015}, o preço para conseguir um novo cliente pode ser até 14 vezes maior que nutrir e manter um que já está na base;
\item A tecnologia no atendimento ao cliente resultou em impessoalidade, com cada vez mais robôs atendentes no lugar de pessoas.
\end{itemize}

Uma das maneiras utilizadas para realizar a comunicação com o cliente é através do Sistema de Atendimento ao Consumidor - mais conhecido como SAC - de cada empresa. O SAC pode ser visto como uma ferramenta para solucionar a grande demanda de dúvidas, solicitações e reclamações de sua base de clientes, consistindo em uma das possíveis estratégias para desenvolver um bom relacionamento com a base de clientes de uma empresa.

Dados trazidos por \cite{hbrdelightcustomer2010} mostram que 48\% dos consumidores que tiveram experiências negativas com o SAC contam a dez ou mais pessoas; enquanto somente 23\% com experiências positivas contam a dez ou mais pessoas. Passamos da fase em que o atendimento ao cliente é um mero suporte, e ser eficiente nesse processo pode garantir a longevidade do seu negócio.

Alguns problemas encontrados em sistemas de atendimento ao consumidor são:
\begin{itemize}
\item Um novo perfil de consumidor surgiu no mercado (os da Geração Y, ou Millennials). Esse perfil é mais fluído e versátil, exigindo inovação dos profissionais de atendimento \cite{salesforcestateofservice2015};
\item Burocracia gerada por processos e tecnologias engessadas e obsoletas criam fluxos confusos e exaustivos para os consumidores, deixando o SAC no passado e por consequência ineficiente;
\item As empresas não estão nos mesmos canais que os clientes e não conseguem acompanhar as mudanças rápidas que a Internet, as redes sociais e as novas tecnologias provocam nos consumidores todos os dias.
\item O principal meio de atendimento ainda é via telefone (de acordo com \cite{cenariodossacs2015}). O consumidor atual não se encaixa mais nesse perfil. Dados mostram que 58\% dos brasileiros não estão dispostos a esperar mais que 5 minutos por uma resposta ao telefone de um SAC.
\end{itemize}

Com base nas necessidades do mercado de um melhor relacionamento com o cliente, nas deficiências dos atuais sistemas de atendimento e nas limitações técnicas que alguns consumidores enfrentam, decidimos desenvolver um Sistema de Atendimento ao Consumidor que funcione através de transmissão de vídeo utilizando a tecnologia WebRTC.

\section{Motivação}
Observa-se a existência de diversas aplicações \textit{web} voltadas para atendimento ao consumidor, com foco em soluções de \textit{tickets}, análise e categorização dos mesmos. 

Entre elas é possível citar Zendesk\footnote{Disponível em: <https://www.zendesk.com/>. Acesso em 30 jun. 2017.} e Desk.com\footnote{Disponível em: <https://www.desk.com/>. Acesso em 30 jun. 2017.} como estado da arte desse segmento. O mesmo acontece no segmento de programas voltados para conversas através de vídeo. No estado da arte dessa categoria podemos citar appear.in\footnote{Disponível em: <https://www.appear.in/>. Acesso em 30 jun. 2017.}, disponível como aplicação web; Hangouts\footnote{Disponível em: <https://hangouts.google.com/>. Acesso em 30 jun. 2017.}, com versão Web e aplicativo móvel; e o Skype\footnote{Disponível em: <https://www.skype.com/>. Acesso em 30 jun. 2017.}, disponível para Web, \textit{desktop} e dispositivos móveis.

Entretanto a correlação dos dois segmentos é pouco explorado, tanto no âmbito da web, como com programas \textit{desktop}, ou seja, sem a necessidade de instalação de programas e \textit{plugins}.  Para o usuário, a facilidade de acesso e a independência do sistema operacional são vantagens de uma aplicação web em relação à sua versão para uma plataforma específica. 

Essa foi a principal motivação para este projeto: o desenvolvimento de uma aplicação web que faça a gerência de chamadas em vídeo e seja de fácil acesso para qualquer tipo de dispositivo. As chamadas iniciam-se por clientes que necessitam de assistência para o produto oferecido pela empresa, e são respondidas por atendentes designados para esse tipo de trabalho.

\section{Objetivo}
\subsection{Objetivo Geral}
Desenvolver um Sistema de Atendimento ao Consumidor que funcione através de transmissão de vídeo. O sistema deverá dispor de uma área administrativa, onde um profissional responsável por prestar atendimento gerenciará seus chamados.

\subsection{Objetivos Específicos}
Os seguintes objetivos específicos são almejados por este trabalho:
\begin{itemize}
	\item Estudar as soluções existentes para atendimento ao consumidor por chamadas de vídeo e identificar os problemas existentes;
	\item Desenvolver controle de acesso por parte dos atendentes.
	\item Desenvolver uma API baseada no modelo \textit{RESTful} para cadastro de usuários e chamadas através de um servidor HTTP.
	\item Desenvolver um módulo que oferece conexões através de \textit{WebSockets} para: 
    	\begin{itemize}
        	\item{Receber requisições de chamadas em vídeo \textit{realtime}.}
            \item{Realizar o \textit{handshake} necessário para conexão \textit{peer-to-peer} entre navegadores.}
      	\end{itemize}
	\item Implementar interface de aplicação web que suporte atualização em tempo real de componentes sem que seja necessário a atualização da página.
\end{itemize}

\section{Justificativa}
O programa desenvolvido será utilizado para resolver tanto problemas de negócio como problemas técnicos. Auxiliará no gerenciamento de chamadas de vídeo entre consumidores e profissionais de suporte/atendimento. Irá segurar chamadas, atendê-las e encerrá-las, mostrando dados e a localidade do cliente.

O modelo de atendimento por vídeo faz as empresas modernas retomarem a proximidade com os clientes. A tecnologia pode ser utilizada para sermos mais pessoais e alcançarmos uma gama maior de perfis de consumidor, por exemplo: o tipo de consumidor que está acostumado a interagir com o vídeo - meio de comunicação que se tornou uma das principais ferramentas de ajuda e comunicação na nossa sociedade atual. 

De acordo com \cite{customericare2015}, o vídeo é o canal de crescimento mais rápido. O número de chamadas de vídeo subiu de 600 milhões em 2010 para 30 bilhões em 2015. Enquanto isso, 2 bilhões de minutos de chamadas via Skype são feitos todos os dias e 36\% dos consumidores já expressam o desejo de terem suporte por meio de vídeo.

No entanto, a transmissão por vídeo é uma tecnologia relativamente nova em sistemas de atendimento, e somente 0,2\% das empresas prestam atendimento via vídeo (de acordo com a \cite{contactcentredoing2014}). 
Empresas de consultoria, da área da saúde, \textit{call} centers e advogados, por exemplo, já adotaram esse modelo. 

No âmbito jurídico, o vídeo já está sendo utilizado para colher depoimentos em processos. Para áreas de recursos humanos, diversas empresas utilizam a chamada por vídeo para realizar entrevistas com candidatos de outra localidade, sendo que aproximadamente 60\% do gerentes usam esse recurso (\cite{videochatinfographic2013}).

Já a transmissão de vídeo através de \textit{smartphones} e computadores pessoais não é um assunto tão recente. Utilizamos programas como Skype e \textit{plug-ins} de terceiros (ex.: Flash) para realizar chamadas entre dispositivos. Isso gera um \textit{overhead} para o consumidor, que fica a mercê da empresa para saber qual ferramenta utilizar para a chamada. Além disso, o usuário pode não ter em seu \textit{smartphone} o aplicativo ou \textit{plug-in} pré-instalado, e pode não dispor de espaço de armazenamento em seu dispositivo para realizar a instalação do software. O sistema desenvolvido precisará somente de um navegador (Google Chrome, Firefox, etc) em ambos nodos da conversação, ou seja, será suportado pela maioria dos \textit{smartphones}, computadores pessoais e \textit{tablets}. Assim, evitamos a necessidade de o usuário instalar programas adicionais em seu dispositivo.

\section{Metodologia}
Visando o sucesso deste projeto, foi adotada uma metodologia de trabalho que, no primeiro momento, consiste em reuniões periódicas com o orientador do trabalho de conclusão a fim de levantarmos requisitos e necessidades existentes do projeto.

Na segunda etapa foram realizadas análises de relatórios, artigos e pesquisas sobre dificuldades que consumidores têm em relação ao serviços de atendimento ao consumidor de uma grande parcela das empresas brasileiras. Foi observado que grande parte dos desejos e reclamações dos clientes dizem respeito à SACs com possibilidade de chamada em vídeo.

Com essa informação em mãos, uma pesquisa foi feita para encontrar as dificuldades que as empresas têm em fornecer um atendimento em vídeo de qualidade e fácil acesso. Análise feita em conjunto com sistemas utilizados para chamadas de videoconferência.

Para justificar a implementação do sistema foram levantados requisitos unindo as dificuldades tanto dos consumidores quanto das empresas em realizar chamados de assistência técnica bem sucedidos através da transmissão de vídeo. 

A solução encontrada diz respeito a um sistema que:
\begin{itemize}
\item Seja multi-dispositivo, ou seja, que funcione tanto em computadores pessoais como em dispositivos móveis;
\item Multi-plataforma, isto é, que esteja disponível em diversos sistemas operacionais, tanto para computadores (\textit{Windows}, \textit{Linux}, \textit{Mac OS}) quanto para dispositivos móveis (\textit{iOS}, \textit{Android});
\item Fácil ou nenhuma instalação, sem necessidade de instalação de aplicativos proprietários;
\item Possua gerenciamento de chamadas.
\end{itemize}

Construído o modelo e as ideias da aplicação, o projeto de implementação foi desenvolvido. O projeto constitui-se de análise de requisitos, descrição de tecnologias e arquitetura, junto das suas funcionalidades.

Com o projeto de implementação em mãos, o desenvolvimento do sistema será feito através de uma plataforma chamada \textit{NodeJS}, utilizando dois \textit{frameworks}: \textit{Express.js} e \textit{Socket.IO}, o primeiro responsável pela API e o segundo para conexão através de \textit{WebSockets}. O  servidor que fornece conexão por \textit{sockets} será responsável por realizar o estabelecimento da conexão entre dois nodos da rede, gerenciar chamadas e a API por guardar informações importantes sobre as mesmas.

Após a conexão entre duas partes a transmissão de mídia (principalmente vídeo) será realizada por uma tecnologia moderna chamada \textit{WebRTC}, especializada em transmissão de mídia em tempo real. 

Por fim, a publicação de todo sistema será feito através do \textit{Heroku}, uma empresa de \textit{PaaS} que fornece toda assistência para fornecer as aplicações online de forma simples e eficiente.

\section{Organização do Texto}
Conhecendo os objetivos, metodologia e a motivação por trás deste projeto, os próximos capítulos dão início às etapas de trabalho propostas.

Apresenta-se no capítulo seguinte a fundamentação teórica: descrições sobre os conceitos teóricos que cercam o sistema a ser implementado, descrição sobre conceitos utilizados por outras aplicações do mesmo ramo e a evolução da tecnologia que foi usada como base na construção do projeto (WebRTC); 

O terceiro capítulo contém uma análise das ferramentas existentes tanto na categoria de atendimento ao consumidor, como na de videoconferência, e uma breve discussão sobre os pontos de melhoria em comparação com este projeto.

No quarto capítulo é descrita a analise de requisitos do software, ou seja, as funcionalidades que serão implementadas no mesmo. Junto dos requisitos serão demonstradas a modelagem e a arquitetura do projeto.

A implementação do projeto encontra-se no quinto capítulo, no qual serão descritas a análise de requisitos, as decisões e descrições de tecnologias e ferramentas utilizadas, a arquitetura escolhida e as funcionalidades do sistema.
