% ----------------------------------------------------------
% Conclusão
% ----------------------------------------------------------
\chapter[Considerações finais]{Considerações finais}\label{chap:conclusao}

Realizar esse trabalho proporcionou um aprendizado importante sobre como enxergar e arquitetar a partir de uma visão geral até entrar em detalhes de implementação, tudo com um objetivo em mente que foi melhorar a situação das empresas perante seus clientes proporcionando formas mais fáceis de conectar ambas partes. Uma aplicação com diversas partes conectadas (cliente, empresa, atendente) necessita de diferentes áreas de conhecimento por isso pode-se dizer que o projeto foi multi-disciplinar.

Como explicação, foi preciso estar em contato com aplicações de gerência de informações, como banco de dados. Descobrir uma maneira de distribuir essas informações em diferentes aplicações, percorrendo áreas como protocolos de rede, conexão entre processos, arquitetura REST. Crucial para esse projeto foi o conceito de informações em tempo real, realizada com web sockets. E por fim como mostrar todo esse tipo de dado através de uma interface HTML. 

Para versões futuras do projeto é recomendado a extensão desse projeto com ferramentas de análise de dados, painéis que coletem mais informações sobre os clientes, suas empresas e seus problemas. Dessa maneira é possível coletar e sintetizar um sistema \textit{self-service}.